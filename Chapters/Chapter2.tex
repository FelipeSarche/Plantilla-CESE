\chapter{Introducción específica} % Main chapter title

\label{Chapter2}

%----------------------------------------------------------------------------------------
%	SECTION 1
%----------------------------------------------------------------------------------------
Este capítulo explica las tecnologías aplicadas en el desarrollo del panel de mensajes variable. Se presentan los requerimientos identificandos al iniciar el trabaja para cumplir con los objetivos.

\section{Estilo y convenciones}
\label{sec:ejemplo}

\subsection{Uso de mayúscula inicial para los título de secciones}

Si en el texto se hace alusión a diferentes partes del trabajo referirse a ellas como capítulo, sección o subsección según corresponda. Por ejemplo: ``En el capítulo \ref{Chapter1} se explica tal cosa'', o ``En la sección \ref{sec:ejemplo} se presenta lo que sea'', o ``En la subsección \ref{subsec:ejemplo} se discute otra cosa''.

Cuando se quiere poner una lista tabulada, se hace así:

\begin{itemize}
	\item Este es el primer elemento de la lista.
	\item Este es el segundo elemento de la lista.
\end{itemize}

Notar el uso de las mayúsculas y el punto al final de cada elemento.

Si se desea poner una lista numerada el formato es este:

\begin{enumerate}
	\item Este es el primer elemento de la lista.
	\item Este es el segundo elemento de la lista.
\end{enumerate}
\section{Requerimientos}
En esta sección se indican los requerimientos que fueron propuestos para el proyecto.

\subsection{Grupos de requerimientos asociados con hardware}

Tarjeta de distribución:
\begin{itemize}
\item Debe tener conectores de entrada IDC40 con la misma distribución de pines usados en las salidas GPIO de la board DE1-SoC.
\item Debe cambiar los niveles de voltaje de 3.3 v a 5 v de todas las entradas. 
\item Debe manejar señales de una frecuencia de entre 5 Mhz a 10 Mhz.
\item Debe tener conectores de salida IDC16 compatibles con la distribución de pines usados en las matrices de leds.
\end{itemize}
Matriz de leds:
\begin{itemize}
\item Debe tener un conector de entrada y un conector de salida IDC16.
\item Debe ser capaz de manejar frecuencias de hasta 10 Mhz.
\item Debe tener drivers de leds con escala de grises de 16 bits.
\item Debe tener 16 leds de alto por 16 leds de ancho.
\item Debe ser capaz de pasar la información de la entrada a la salida.
\item Debe poder realizarse control por barrido. 
\item La pantalla tendrá una dimensión de 400 pixeles de largo por 96 pixeles máximo.
\end{itemize}

\subsection{Grupos de requerimientos asociados con el software}

Sistema operativo:
\begin{itemize}
\item Debe ser capaz de inicial y restablecer el programa del procesador automáticamente.
\item Debe ser capaz de conectarse a la red lan. 
\item Debe ser capaz de compartir una carpeta para el proceso de cargado de imágenes.
\item Almacenamiento interno de hasta 500 imágenes.
\end{itemize}

Firmware procesador:
\begin{itemize}
\item Debe ser capaz de leer las imágenes que se encuentran en la carpeta compartida.
\item Debe ser capaz de escribir en una  memoria ram accesible para el FPGA  las imágenes a ser desplegadas.
\item Debe administrar el despliegue de las imágenes por medio de un archivo externo que sea enviado desde la central de control.
\end{itemize}
\subsection{Grupos de requerimientos asociados al FPGA}
FPGA:
\begin{itemize}
\item El FPGA debe ser capaz de manejar a la pantalla led a una frecuencia de  30 hz.
\item Debe leer la información del los leds de una memoria ram que comparta con el procesador.
\item Debe manejar los drivers de led a través de una máquina de estado finita.   
\end{itemize}